\chapter{Conclusion}\label{ch:conclusion}

In this dissertation, I challenge common RC design rules, and
demonstrate increasingly simple RCs that break these rules and still
perform well.

In \cref{ch:low-connectivity}, I demonstrate ESN-based RCs with five
classes of simple internal connectivity that all perform well on the
system forecasting task. Despite the fact that recurrence is a common
requirement in ESN design, two of these classes of connectivity have
\emph{no recurrence}, but continue to perform well. A network with only a
single internal cycle, or no cycles at all, can perform as well as a
network with many recurrent cycles. These simpler reservoir structures
may benefit hardware-based RCs, where every internal connection has an
associated cost.

This work reveals some new lines of possible future
research. Primarily, the $\tilde{\epsilon}$ metric is a more valuable
measure than $\epsilon_1$ of how well the ESN reproduces the climate
of the system, but ultimately still fails to measure attractor
similarity directly. The development of such a metric would be very
important to the field, as it begins to consider how well an RC
reproduces climate in addition to short-term forecasting.

One of the simple connectivity classes I explore is a simple line
network. The fact that this line network can perform as well as a
full-connectivity network indicates that perhaps the reservoir's role
is to produce a variety of time delays for the RC output layer to draw
on. In \cref{ch:nvar}, I summarize the NVAR method, which parallels
the RC method but instead takes time-delay taps of the input directly
rather than relying on a reservoir. I also sketch a new proof in the
field of reservoir computing that shows a certain type of
output-nonlinear ESN is equivalent to an infinite-tap NVAR.

In \cref{ch:nvar-application} I build on this proof with my
collaborators to implement NVARs that solve the traditional RC tasks
of system state inference and system forecasting. I show
that the NVAR can solve these tasks with only a few, finite taps, and
that the specific nonlinearity used in the proof can be modified an
extended in obvious ways to expand the number of problems it can
solve. This opens the door to an entire expanding tree of interesting
research. Now any task previously solved in an RC context can be
attempted in an NVAR context, and the success or failure of the NVAR
at solving a given problem will contribute meaningfully to both the
NVAR and RC fields. In particular, the simplicity of the NVAR approach
lends itself to hardware implementation, and avoids the complexities
of hardware-based RCs.

Applying the NVAR approach to Mackey-Glass system forecasting reveals
possibly the most important avenue for future research. Cubic terms
are not sufficient for all problems, but adding higher-order monomials
directly to the NVAR leads to an exponential explosion in training
time. Future research will need to resolve this by including these
higher-order terms without exhaustively including all monomials, and
may possibly make use of nonlinear functions with infinite power
series.

The NVAR is a reservoir computer pruned down to its most essential
components, and now that I know it is possible build a high-performance reservoir computer
without a reservoir \emph{at all}, I am excited to see what new
directions the RC field moves towards.
