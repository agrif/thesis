\chapter{Low-Connectivity Reservoirs}\label{ch:low-connectivity}

% how to choose hyperparameters? grid search? gradient descent?
% introduce our task: chaotic system forecasting, using \tilde{\epsilon}

\section{Bayesian Optimization}
% overview, parameter range, focus on k.
% during opt, will often find k=1

\section{Structure of Low-Connectivity Reservoirs}
% disconnected components
% k=1 single cycle
% what if cut the cycle?
% what if it's *all loops*

\section{Training, Forecasting, and Evaluation}
% use mean 0 variance 1 signals from systems ...
% length of time periods
% use cross-validation for alpha
% discussion of \tilde{\epsilon} vs \epsilon

\section{RC Symmetries and their Consequences}
% RC forecast symmetry
% not shared by system, problem for output
% plot!

\section{Results}
% detailed results for lorenz, violin plots, attractors
% for other systems

\section{Cross-task Performance}
% optimized Lorenz used on double-scroll

\section{Conclusion}
% optimization is useful, and can yield surprising results
% \tilde{\epsilon} is good
% rho_r = 0, k=1 are useful (weird!)
% but: they are rarer. This might be useful
% NARX note: also look forward to next chapter. what unifies these?
% line reservoir is almost just some time delay taps
% can this generalize to many tasks? can it make hardware simpler?
